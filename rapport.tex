\documentclass{article}
\usepackage{fullpage}
\usepackage{mathrsfs}
\usepackage[french]{babel}
\usepackage[utf8]{inputenc}
\usepackage{geometry}
\geometry{a4paper,hmargin=2cm,vmargin=1.6cm}
\usepackage{amsmath,amssymb}
\usepackage{amsthm}

\usepackage{listings}

\newtheorem{theorem}{Théorème}
\newtheorem{corollaire}[theorem]{Corollaire}
\newtheorem{lemma}[theorem]{Lemme}
\theoremstyle{definition}
\newtheorem{definition}{Définition}
\newtheorem{exemple}{Exemple}
\newtheorem{remarque}{Remarque}
\newtheorem{axiome}{Axiome}
\renewcommand{\leq}{\leqslant}
\renewcommand{\preceq}{\preccurlyeq}

\usepackage{graphicx}
\usepackage{tikz}

\author{Marianne Déglise}
\title{Projet Programmation 1 L3\\
Exercice 1}
\date{}

\renewcommand{\emptyset}{\varnothing}

\begin{document}

\maketitle

\begin{section}{Module my\_list}

On réimplémente les fonctions de bases de la librairie Liste Ocaml avec notre propre module de liste (en devinant quelles fonctions implémenter grâce aux types des fonctions de my\_list.mli.


\begin{lstlisting}[language=caml]
type 'a my_list =
  |Nil
  |Cons of 'a * 'a my_list

let string_of_list str_fun u =
  let rec string_content = function
    |Nil -> ""
    |Cons (x, Nil) -> (str_fun x)
    |Cons (x, v) -> (str_fun x) ^ ", " ^ (string_content v)
  in "[" ^ (string_content u) ^ "]" 

let hd u =
  match u with
    |Nil -> None
    |Cons (x, xs) -> Some x

let tl = function
  |Nil -> None
  |Cons (x, xs) -> Some xs

let rec length u =
  match u with
    |Nil -> 0
    |Cons (x,xs) -> 1 + (length xs)

let rec map funab u =
  match u with
    |Nil -> Nil
    |Cons (x,xs) -> Cons(funab x, map funab xs)
\end{lstlisting}

\end{section}

\newpage

\begin{section}{Nouvelle version de test}

On adapte le code de test\_list.ml pour que la fonction travaille sur notre nouveau type my\_list. A chaque fois qu'on faisait appel à des fonctions du module List, on remplace par My\_list, et on introduit tous les Nil et Cons par My\_list. pour signaler qu'ils font partie de ce module.


\begin{lstlisting}[language=caml]
let string_of_list str_fun l = 
  let rec string_content = function
    | My_list.Nil  -> ""
    | My_list.Cons(x, Nil)  -> (str_fun x)
    | My_list.Cons(x, l) -> (str_fun x) ^ ", " ^ (string_content l) 
  in "[" ^ (string_content l) ^ "]" in

let string_of_nat_list = string_of_list string_of_int in
let string_of_string_list = string_of_list (fun x -> x) in

let empty = My_list.Nil in
let one = My_list.Cons("a",My_list.Nil) in 
let lst = My_list.Cons(1, 
My_list.Cons(3,
My_list.Cons(6,
My_list.Cons(10, 
My_list.Cons(15, 
My_list.Cons(21, 
My_list.Cons(28, 
My_list.Cons(36, 
My_list.Cons(45, 
My_list.Cons(55, My_list.Nil)))))))))) in

let test_hd () = 
  Printf.printf "Tete de %s : %s.\n" (string_of_string_list one)
   
  (Option.get (My_list.hd one));
  Printf.printf "Tete de %s : %d.\n\n" (string_of_nat_list lst)
   
  (Option.get (My_list.hd lst))

in let test_tl () = 
  Printf.printf "Queue de %s : %s.\n" (string_of_string_list one)
   (string_of_string_list (Option.get (My_list.tl one)));
  Printf.printf "Queue de %s : %s.\n\n" (string_of_nat_list lst)
   (string_of_nat_list (Option.get (My_list.tl lst)))

in let test_length () = 
  Printf.printf "Taille de %s : %d.\n" (string_of_string_list one) (My_list.length one);
  Printf.printf "Taille de %s : %d.\n" (string_of_nat_list lst) (My_list.length lst);
  Printf.printf "Taille de %s : %d.\n\n" (string_of_string_list empty) 
  (My_list.length empty)

in let test_map ()= 
  Printf.printf "Map de (x -> xx) sur %s : %s.\n" (string_of_string_list one)
   (string_of_string_list (My_list.map (fun s -> s ^ s) one));
  Printf.printf "Map de (x -> 2x) sur %s : %s.\n" (string_of_nat_list lst) 
  (string_of_nat_list (My_list.map (fun n -> 2 * n) lst));
  Printf.printf "Map de (x -> 2x) sur %s : %s.\n\n" (string_of_nat_list empty) 
  (string_of_nat_list (My_list.map (fun n -> 2 * n) empty));

in test_hd(); test_tl(); test_length(); test_map()
\end{lstlisting}

\end{section}

\newpage

\begin{section}{Dépendances avec Makefile}

Pour compiler test\_my\_list, on a besoin d'indiquer sa dépendance de my\_list, que l'on indique dans Makefile. On modifie l'entrée all pour lui faire compiler test\_my\_list et le rapport, et clean pour nettoyer les résultats de la compilation de latex dont on n'a pas besoin.

\begin{lstlisting}[language=caml]
all : test_my_list rapport.pdf

clean:
	rm -rf test_my_list *.cmi *.cmo *~ rapport.pdf rapport.aux rapport.log

test_my_list : my_list.cmo test_my_list.ml
	ocamlc -o test_my_list my_list.cmo test_my_list.ml

my_list.cmo : my_list.cmi my_list.ml
	ocamlc -c my_list.ml

my_list.cmi : my_list.mli
	ocamlc -c my_list.mli


rapport.pdf : rapport.tex
	pdflatex rapport.tex
\end{lstlisting}
\end{section}


\end{document}